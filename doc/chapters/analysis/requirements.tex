\section{Requirements}

\subsection{Functional requirements}

\begin{itemize}
	\item Registration process: new users must register in the system by
		declaring username, password and if they are customers or
		restaurant owners.
	\item The system shall handle login process: with a username and
		password, a user identifies himself/herself within the system,
		so the system is able to manage all the data concerning him/her.
	\item The system shall provide appropriate viewers for the users to see
		list of restaurants available and/or list of reservations of
		customers/restaurant owners.
	\item A costumer should be able to book a table of a restaurant
		choosing from a list, selecting date, hour and number of seats.
	\item The system shall handle the reservation process, managing also the
		residual available seats in each restaurant (in order to avoid
		to take a reservation when the restaurant is full).
	\item A restaurant owner should be able to add, and if necessary modify,
		all the details of his/her restaurant in the list provided to
		customers by the application.
\end{itemize}

\subsection{Non-functional requirements}

\begin{description}
	\item[Availability] The application must be always online, in order to
		be used at any time by users.
	\item[Usability] The application must be easy to use, with a simple and
		intuitive user interface in order to have an early spreading
		among users.
	\item[Portability] The application must be portable, so that any user
		can use it, independently from the system that he/she dispose.
	\item[Data Persistency] The application must use a relational database
		to achieve data persistency.
	\item[Concurrency] The application must handle multiple users at a time.
\end{description}
