\section{Relationships}

\subsection{One-to-one Relationships}

In our case a one-to-one relationship is between a user and a restaurant. In
fact, a restaurant is owned by a user. So the one-to-one relationship is defined
in the \code{User} and \code{Restaurant} classes as follows:

\lstinputlisting[language=Java, label={lst:restauranto2o},
caption={One-to-one relationship in \code{Restaurant\_.java}}]{Restaurant_o2o.java}

\lstinputlisting[language=Java, label={lst:usero2o},
caption={One-to-one relationship in \code{User\_.java}}]{User_o2o.java}

%TODO: clarify
Unlike \code{Restaurant}, in the annotation \code{@OneToOne} in the \code{User}
class we specified the attribute \code{mappedBy}, in fact a bidirectional
relationship brings with it the concept of ``owner side'' and ``reverse side''.
Owner is the \code{Entity} whose report annotation (\code{@OneToOne}) does not
specify the \code{mappedBy}.  In essence, for a bidirectional relationship it is
always the entity linked to the table that holds the relational constraint of
foreign keys.

The reverse side indicates exactly this information (``I am not the Owner, look
at the other the indicated entity'') by inserting in the attribute mappedBy the
name of the relation field in the opposite entity (owner of the Entity
Restaurant).

\subsection{One-to-many / Many-to-one Relationships}

This type of relationship is usually modeled with a foreign key from one table
to another so that different records in one table can reference the same record
in the other. In our case we have a user or a restaurant that can make multiple
reservations.

\lstinputlisting[language=Java, label={lst:usero2m},
caption={One-to-many relationship in \code{User\_.java}}]{User_o2m.java}
\lstinputlisting[language=Java, label={lst:restauranto2m},
caption={One-to-many relationship in \code{Restaurant\_.java}}]{Restaurant_o2m.java}
\lstinputlisting[language=Java, label={lst:reservationm2o},
caption={Many-to-one relationship in \code{Reservation\_.java}}]{Reservation_m2o.java}

Also in this case we use \code{@JoinColumn} (with \code{@ManytoOne}) to specify
the \code{Reservation}'s columns that hold the foreign keys of User and
Restaurant respectively.

In the \code{User} and \code{Restaurant} classes we use \code{@OneToMany} with
the \code{mappedBy} attribute to define in which field the relationship with
Reservation occurs.

\subsection{Many-to-many Relationships}

There is also a \code{@ManyToMany} annotation that can be used to represent
many-to-many relationships. This annotation is used in conjunction with the
\code{@JoinTable} annotation that specifies the name of the table used for the
relationship along with the join columns:

\lstinputlisting[language=Java, label={lst:m2m}, caption={Many-to-many
annotations}]{m2m.java}
