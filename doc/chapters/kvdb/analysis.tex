\section{Relational Data Model Analysis}

The current data model is the one described in the Figure~\ref{fig:er}.

\subsection{Volume Table}

The first step towards a fair analysis is dimensioning the problem. To do so we
have to create a volume table.

\vspace{0.3cm}
\resizebox{\textwidth}{!}{%
	\begin{tabular}{|l|l|l|l|}
		\hline
		\thead{Name} & \thead{E/R} & \thead{Instances} & \thead{Justification} \\
		\hline
		User & E & \(50\,000\) & Initial Assumption \\
		\hline
		Owning & R & \(200\) & We suppose that 2\% of users is a restaurant owner \\
		\hline
		Restaurant & E & \(200\) & \makecell{Every restaurant owner has only 1
			restaurant\\ (Cardinality (1,1))} \\
		\hline
		Reservation & R & \(1\,500\,000\) & \makecell{On average a user has 30 reservations per year,\\ and that an owner in his restaurants has on average\\ 20 reservation per day (7300 per year).}\\
		\hline
	\end{tabular}
}

\subsection{Possible Operations}

After dimensioning the problem we provide a set of CRUD operations that can be
performed on the database. Each operations has a frequency defined as the number
of times the operation is performed per day.

\subsubsection{Login}

\begin{itemize}
	\item Description: Find a user in the system.
	\item Frequency: \(35\,000\) per week.
\end{itemize}

\subsubsection{Register}

\begin{itemize}
	\item Description: Insert a new user.
	\item Frequency: \(100\) per week.
\end{itemize}

\subsubsection{List of Restaurants}

\begin{itemize}
	\item Description: A customer request the list of all the available
		restaurants.
	\item Frequency: \(175\,000\) per week (5 times per login session on
		average).
\end{itemize}

\subsubsection{List of Reservations per User}

\begin{itemize}
	\item Description: Find all the reservation of a given user.
	\item Frequency: \(175\,000\) per week (Considering after a login and after each booking we ask for a new list of reservation this operation is performed 5 times per week for each user).
\end{itemize}

\subsubsection{List of Reservations per Restaurant}

\begin{itemize}
	\item Description: Find all the reservations for a given restaurant.
	\item Frequency: \(140\,000\) per week (every owner will check multiple times the list of reservation (10 times per day) at his restaurant we assume on average 70 per week for each user).
\end{itemize}

\subsubsection{Modify a Restaurant}

\begin{itemize}
	\item Description: Update the informations of a given restaurant.
	\item Frequency: \(25\) per week (we assume that these informations are
		not so likely to change frequently).
\end{itemize}

\subsubsection{Book a table in a Restaurant}

\begin{itemize}
	\item Description: Insert a reservation.
	\item Frequency: \(105\,000\) per week.
\end{itemize}


\subsubsection{Check available seats in a Restaurant}

\begin{itemize}
	\item Description: Find the number of seats available in a restaurant.
	\item Frequency: \(210\,000\) per week (considering that a customer will perform on average 3 reservation per week, may happen that some restaurant are full or a customer just want to know if there are seats available, so there will be slightly more check then reservations).
\end{itemize}

\subsubsection{Delete a Reservation}

\begin{itemize}
	\item Description: Remove a reservation from the database.
	\item Frequency: \(525\) per week (0,5\% of weekly reservations).
\end{itemize}

\subsubsection{Add a Restaurant}

\begin{itemize}
	\item Description: Add a restaurant in the system.
	\item Frequency: \(3\) per week.
\end{itemize}


\subsection{Analysis Results}

The most important thing that we can see from these analysis is that the
database is used mostly for read operations. The prominent operations are:

\begin{itemize}
	\item List of Restaurant (175\,000 access/week)
	\item List of Reservation per User (175\,000 access/week)
	\item List of Reservation per Restaurant (140\,000 access/week)
	\item Check available seats in a Restaurant (210\,000 access/week)
\end{itemize}

Starting from this situation we must consider the complexity in terms of join
operations that must be performed for each operation.

\subsubsection{List of Restaurants}

Read all the instances in restaurants. It is not necessary any join operation.

\subsubsection{List of Reservation per User}

In this operation we need a join between Restaurant and Reservation tables.  We
need the name of each Restaurant where the User with given userID have a
Reservation.

\subsubsection{List of Reservation per Restaurant}

In this operation we need a join between User and Reservation tables.  We need
the name of each user that have a Reservation in the Restaurant with given
restaurantID.

\subsubsection{Check available seats in a Restaurant}

In this operation we need a join between Restaurant and Reservation table. We
need the number of remaining seats in a given restaurant for the given date and
hour.

The first operation, even if it's a frequently used read operation, it doesn't
require any join.  Instead, the last three operations are \textbf{frequently}
used read operations that need a join. They are the ones that most will achieve
advantages in terms of performances using a Key-Value Database among the above
listed ones. In particular, the key-value database will be used as a cache of
the relational database for the aforementioned read operations.

\subsection{Data Model Translation}

In this step we have to create a key value data model that fits our needs.  In
other words we need a key-value data model that can be derived from the join of
three tables involved in the previously mentioned operations.

Starting from the ER model in Figure~\ref{fig:er} we can translate our join
table as follow:

\vspace{0.75cm}

\centerline{\textbf{reservations:<userId>:<restaurantId>:<date>:<time>:<attribute name>=value}}

\vspace{0.75cm}

Attribute name can be:
\begin{itemize}
	\item id
	\item seats
	\item user\_username
	\item restaurant\_name
\end{itemize}

\vspace{0.75cm}
The introduction of the fields <date> and <time> in the key give us an advantage 
when we need to check for available seats. In fact when we need this operation
we just have to iterate on the reservations with given date and time without the
need to check for that. 

\vfill
