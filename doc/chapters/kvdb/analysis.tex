\section{Relational Data Model Analysis}

The current data model is the one described in the Figure~\ref{fig:er}.

\subsection{Volume Table}

The first step towards a fair analysis is dimensioning the problem. To do so we
have to create a volume table.

\vspace{0.3cm}
\resizebox{\textwidth}{!}{%
	\begin{tabular}{|l|l|l|l|}
		\hline
		\thead{Name} & \thead{E/R} & \thead{Istances} & \thead{Justification} \\
		\hline
		User & E & \(50\,000\) & Initial Assumption \\
		\hline
		Owning & R & \(1\,000\) & We suppose that 2\% of users is a restaurant owner \\
		\hline
		Restaurant & E & \(1\,000\) & \makecell{Every restaurant owner has only 1
			restaurant\\ (Cardinality (1,1))} \\
		\hline
		Reservation & R & \(500\,000\) & On average a user has 10 reservations \\
		\hline
	\end{tabular}
}

\subsection{Possible Operations}

After dimensioning the problem we provide a set of CRUD operations that can be
performed on the database. Each operations has a frequency defined as the number
of times the operation is performed per day.

\subsubsection{Login}

\begin{itemize}
	\item Description: Find a user in the system.
	\item Frequency: \(5\,000\) per day.
\end{itemize}

\subsubsection{Register}

\begin{itemize}
	\item Description: Insert a new user.
	\item Frequency: \(250\) per day.
\end{itemize}

\subsubsection{List of Restaurants}

\begin{itemize}
	\item Description: A customer request the list of all the available
		restaurants.
	\item Frequency: \(10\,000\) per day (2 times per login session on
		average).
\end{itemize}

\subsubsection{List of Reservations per User}

\begin{itemize}
	\item Description: Find all the reservation of a given user.
	\item Frequency: \(10\,000\) per day (2 times per login session on
		average).
\end{itemize}

\subsubsection{List of Reservations per Restaurant}

\begin{itemize}
	\item Description: Find all the reservations for a given restaurant.
	\item Frequency: \(5\,000\) per day (every owner will check multiple times
		the list of reservation at his restaurant we assume on average 5
		times per owner).
\end{itemize}

\subsubsection{Modify a Restaurant}

\begin{itemize}
	\item Description: Update the informations of a given restaurant.
	\item Frequency: \(25\) per day (we assume that these informations are
		not so likely to change frequently).
\end{itemize}

\subsubsection{Book a table in a Restaurant}

\begin{itemize}
	\item Description: Insert a reservation.
	\item Frequency: \(2\,500\) per day.
\end{itemize}


\subsubsection{Check available seats in a Restaurant}

\begin{itemize}
	\item Description: Find the number of seats available in a restaurant.
	\item Frequency: \(3\,500\) per day (may happen that some restaurant are
		full or a customer just want to know if there are seats
		available, so there will be slightly more check then
		reservations).
\end{itemize}

\subsubsection{Find a given Restaurant}

\begin{itemize}
	\item Description: Find the information about a given restaurant.
	\item Frequency: \(2\,000\) per day.
\end{itemize}


\subsubsection{Delete a Reservation}

\begin{itemize}
	\item Description: Remove a reservation from the database.
	\item Frequency: \(50\) per day.
\end{itemize}

\subsubsection{Add a Restaurant}

\begin{itemize}
	\item Description: Add a restaurant in the system.
	\item Frequency: \(50\) per day.
\end{itemize}

\subsection{Access Table}

After dimensioning the number of instances we will have for each Entity and
Relation, and hypothesized the frequency of each operation that can be performed
on the database we can analyze the amount of operations needed to the
accomplishment of these operations in order to recognize which are the most
expensive. Elementary operations are write and read.

We assume that a write operation is always preceeded from a read operation, this
means that when the system is performing a write it's indeed doing two
elementary operations.

\subsubsection{Login}

\begin{tabular}{|l|l|l|}
	\hline
	\thead{\# Elementary\\operations} & \thead{Operation\\Type} & \thead{Justification} \\
	\hline
	\(1\) & Read & Read once in User.\\
	\hline
\end{tabular}

Number of elementary operations needed for a single operation: \(1\).

Number of elementary operations per day: \(1 \times 5\,000 = 5\,000\).

\subsubsection{Register}

\begin{tabular}{|l|l|l|}
	\hline
	\thead{\# Elementary\\operations} & \thead{Operation\\Type} & \thead{Justification} \\
	\hline
	\(2 \times 1\) & Write & Write once in User.\\
	\hline
\end{tabular}

Number of elementary operations needed for a single operation: \(2\).

Number of elementary operations per day: \(2 \times 250 = 500\).

\subsubsection{List of Restaurants}

\begin{tabular}{|l|l|l|}
	\hline
	\thead{\# Elementary\\operations} & \thead{Operation\\Type} & \thead{Justification} \\
	\hline
	\(1000\) & Read & Read all the instances in Restaurant.\\
	\hline
\end{tabular}

Number of elementary operations needed for a single operation: \(1\,000\).

Number of elementary operations per day: \(1\,000 \times 10\,000 = 10\,000\,000\).

\subsubsection{List of Reservation per User}

\resizebox{\textwidth}{!}{%
	\begin{tabular}{|l|l|l|}
		\hline
		\thead{\# Elementary\\operations} & \thead{Operation\\Type} & \thead{Justification} \\
		\hline
		\(1\) & Read & Read once in User.\\
		\(1 \times 10\) & Read & \makecell{Read 10 times in Reservation (From\\volume table we know that a user has\\on average 10 reservations)}\\
		\(1 \times 10\) & Read & \makecell{Read 10 times in restaurant to\\ retrieve restaurant name}\\
		\hline
	\end{tabular}
}

Number of elementary operations needed for a single operation: \(21\).

Number of elementary operations per day: \(21 \times 10\,000 = 210\,000\).

\subsubsection{List of Reservation per Restaurant}

\resizebox{\textwidth}{!}{%
	\begin{tabular}{|l|l|l|}
		\hline
		\thead{\# Elementary\\operations} & \thead{Operation\\Type} & \thead{Justification} \\
		\hline
		\(1\) & Read & Read once in Restaurant.\\
		\(1 \times 500\) & Read & \makecell{Read 500 times in Reservation (From\\ volume table we know that a Restaurant\\ has on average 500 reservations)}\\
		\(1 \times 500\) & Read & \makecell{Read 500 times in User to\\ retrieve username}\\
		\hline
	\end{tabular}
}

Number of elementary operations needed for a single operation: \(1\,001\).

Number of elementary operations per day: \(1\,001 \times 5\,000 = 5\,005\,000\).

\subsubsection{Modify a Restaurant}

\begin{tabular}{|l|l|l|}
	\hline
	\thead{\# Elementary\\operations} & \thead{Operation\\Type} & \thead{Justification} \\
	\hline
	\(2 \times 1\) & Write & Write once in Restaurant.\\
	\hline
\end{tabular}

Number of elementary operations needed for a single operation: \(2\).

Number of elementary operations per day: \(2 \times 25 = 50\).

\subsubsection{Book a table in a restaurant}

\begin{tabular}{|l|l|l|}
	\hline
	\thead{\# Elementary\\operations} & \thead{Operation\\Type} & \thead{Justification} \\
	\hline
	\(1\) & Read & Read once in Restaurant.\\
	\hline
\end{tabular}

Number of elementary operations needed for a single operation: \(1\).

Number of elementary operations per day: \(1 \times 2\,500 = 2\,500\).

\subsubsection{Check available seats in a Restaurant}

\begin{tabular}{|l|l|l|}
	\hline
	\thead{\# Elementary\\operations} & \thead{Operation\\Type} & \thead{Justification} \\
	\hline
	\(1\) & Read & Read once in Restaurant.\\
	\hline
\end{tabular}

Number of elementary operations needed for a single operation: \(1\,001\).

Number of elementary operations per day: \(1 \times 3\,500 = 3\,500\).

\subsubsection{Find a given Restaurant}

\begin{tabular}{|l|l|l|}
	\hline
	\thead{\# Elementary\\operations} & \thead{Operation\\Type} & \thead{Justification} \\
	\hline
	\(1\) & Read & Read once in Restaurant.\\
	\hline
\end{tabular}

Number of elementary operations needed for a single operation: \(1\).

Number of elementary operations per day: \(1 \times 2\,000 = 2\,000\).

\subsubsection{Delete a Reservation}

\begin{tabular}{|l|l|l|}
	\hline
	\thead{\# Elementary\\operations} & \thead{Operation\\Type} & \thead{Justification} \\
	\hline
	\(2 \times 1\) & Write & Write once in Reservation.\\
	\hline
\end{tabular}

Number of elementary operations needed for a single operation: \(2\).

Number of elementary operations per day: \(2 \times 50 = 100\).

\subsubsection{Add a Restaurant}

\begin{tabular}{|l|l|l|}
	\hline
	\thead{\# Elementary\\operations} & \thead{Operation\\Type} & \thead{Justification} \\
	\hline
	\(2 \times 1\) & Write & Write once in Restaurant.\\
	\hline
\end{tabular}

Number of elementary operations needed for a single operation: \(2\).

Number of elementary operations per day: \(2 \times 50 = 100\).

\subsection{Analysis Results}

The most important thing that we can see from these analysis is that the
database is used mostly for read operations. The prominent operations are in
fact the List of Restaurants (\(10\,000\,000\) accesses per day) and the List of
Reservation per Restaurant (\(5\,005\,000\) accesses per day).

This situation lead us to evaluate the possibility to adopt an hybrid solution.
The existent relational database is maintained and will be used only for the
writes, granting the consistency of the data. For the read operations we will
use instead a Key-Value database. These databases are way faster then the
relational ones, and will grant a faster response of the system to the large
amount of reads that will be issued daily.

This type of solution need that the two database must be mutually consistent, so
when a write operation will be performed on the relational database, this update
must be propagated to the other one in an atomic way.

\subsection{Data Model Translation}

The first step towards the choosen solution is to translate the relational data
model to a key-value one.

Starting from the ER model in Figure~\ref{fig:er} we can translate our entities
as follow:

\subsubsection{User}

The User table in the ER model is implemented in this way:

\begin{tabular}{|l|l|l|}
	\hline
	\thead{UserID} & \thead{Username} & \thead{Password} \\
	\hline
\end{tabular}

The corresponding key-value translation is in the form:

\centerline{\textbf{users:<userid>:<attributeName> = <value>}}

More precisely we have:

\centerline{\textbf{users:<userid>:username = <value>}}

\centerline{\textbf{users:<userid>:password = <value>}}

\subsubsection{Restaurant}

The Restaurant table similarly can be translated from:

\resizebox{\textwidth}{!}{%
	\begin{tabular}{|l|l|l|l|l|l|l|l|l|l|}
		\hline
		\thead{RestaurantID} & \thead{UserID} & \thead{Name} & \thead{Genre} &
		\thead{Cost} & \thead{City} & \thead{Address} & \thead{Description} &
		\thead{Seats} & \thead{OpeningHours} \\
		\hline
	\end{tabular}
}

To the key-value representation:

\centerline{\textbf{restaurants:<restaurantid>:attributeName = <value>}}

\subsubsection{Reservation}

The last table of the ER model come out from a many to many relationship, and is
implemented as follow:

\begin{tabular}{|l|l|l|l|l|l|}
	\hline
	\thead{ReservationID} & \thead{UserID} & \thead{RestaurantID} &
	\thead{Date} & \thead{Hour} & \thead{Seats} \\
	\hline
\end{tabular}

And it will be translated in the following key-value representation:

\centerline{\textbf{reservations:<reservationid>:<attributeName> = <value>}}

In this way we will threat as attribute both the \code{UserID} and the
\code{Restau\-rantID}. In this way the key value data model will be more storage
efficient, since we can use the same ``primary key'' (the prefix that uniquely
identifies the attributes of a same reservation) both to retrieve the list of
reservation filtering them by \code{RestaurantID} or by \code{UserID}. This by
the way lead to a slightly higher complexity on performing queries to retrieve
the desired values.
