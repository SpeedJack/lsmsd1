\chapter{EntityManager}

JPA also defines an \code{EntityManager}, whose purpose is the runtime
management of queries and transactions on persistent objects.

An \code{EntityManager} instance is associated with a persistence context, and
it is used to interact with the database. A persistence context is a set of
entity instances, which are actually the objects or instances of the model
classes.

The \code{EntityManager} is used to manage entity instances and their life
cycle, such as create entities, entities, remove entities, find and query
entities.

An \code{EntityManager} instance can be created as described below, using
\code{EntityManagerFactory}.

\lstinputlisting[language=Java, label={lst:em},
caption={\code{EntityManager.java}}]{EntityManager.java}

The class shown in Listing~\ref{lst:em} is a wrapper class for the Hibernate's
\code{EntityManager}. In our example we have created three
\code{EntityManager}'s child classes for each entity of our system:
\code{UserManager}, \code{RestaurantManager} and \code{ReservationManager}.
